\documentclass{article}
\usepackage[T1]{fontenc}
\usepackage[utf8]{inputenc}
\usepackage[a4paper]{geometry}
\usepackage{graphicx}
\geometry{verbose,tmargin=2cm,bmargin=2cm,lmargin=1.5cm,rmargin=1.5cm}
\usepackage[usenames,dvipsnames]{xcolor}
\usepackage{url}
\usepackage{hyperref}
\hypersetup{
    bookmarks=true,         % show bookmarks bar?
    unicode=true,          % non-Latin characters in Acrobat’s bookmarks
    bookmarksnumbered=false,
    bookmarksopen=false,
    breaklinks=true,
    backref=true,
    pdftoolbar=true,        % show Acrobat’s toolbar?
    pdfmenubar=true,        % show Acrobat’s menu?
    pdffitwindow=false,     % window fit to page when opened
    pdfstartview={FitH},    % fits the width of the page to the window
    pdftitle={Partial discharge analysis and clustering}
    pdfauthor={Oscar Perpiñán Lamigueiro},     % author
    pdfsubject={Partial discharge clustering},   % subject of the document
    pdfcreator={AucTeX/Emacs},   % creator of the document
    pdfproducer={LaTeX}, % producer of the document
    pdfkeywords={partial discharge, clustering, feature generation,
      graphical tools}, % list of keywords
    pdfnewwindow=true,      % links in new window
    pdfborder={0 0 0},
    colorlinks=true,       % false: boxed links; true: colored links
    linkcolor=BrickRed,          % color of internal links
    citecolor=BrickRed,        % color of links to bibliography
    filecolor=black,      % color of file links
    urlcolor=Blue           % color of external links 
}

\makeatletter
%%%%%%%%%%%%%%%%%%%%%%%%%%%%%% Textclass specific LaTeX commands.
\usepackage[noae]{Sweave}
\newcommand{\Rcode}[1]{{\texttt{#1}}}
\newcommand{\Robject}[1]{{\texttt{#1}}}
\newcommand{\Rcommand}[1]{{\texttt{#1}}}
\newcommand{\Rfunction}[1]{{\texttt{#1}}}
\newcommand{\Rfunarg}[1]{{\textit{#1}}}
\newcommand{\Rpackage}[1]{{\textit{#1}}}
\newcommand{\Rmethod}[1]{{\textit{#1}}}
\newcommand{\Rclass}[1]{{\textit{#1}}}

%%%%%%%%%%%%%%%%%%%%%%%%%%%%%% User specified LaTeX commands.
%\VignetteIndexEntry{Partial Discharges Clustering}
\usepackage{flafter}
\usepackage{boxedminipage}
\renewenvironment{Schunk}{\begin{center}
    \scriptsize
    \begin{boxedminipage}{0.95\textwidth}}{
    \end{boxedminipage}\end{center}}

\usepackage{siunitx}
\sisetup{per=fraction,fraction=nice, decimalsymbol=comma}
%\usepackage{lscape}
\usepackage{mathpazo}%Letra palatino con fuentes para matemáticas

\makeatother

\begin{document}

\setkeys{Gin}{width=0.5\textwidth}

\title{\texttt{pdCluster}: Partial Discharges Clustering} 


\author{Oscar Perpiñán Lamigueiro \and Miguel Angel Sánchez Urán}


\date{17 May 2011}

\maketitle

\begin{Schunk}
\begin{Sinput}
> library(pdCluster)
\end{Sinput}
\end{Schunk}

The set of examples will use dataset which is loaded with:
\begin{Schunk}
\begin{Sinput}
> load("~/Investigacion/PD/Datos/20100922/DescargasRAW.RData")
\end{Sinput}
\end{Schunk}
\section{Feature generation}

\subsection{Prony}
\label{sec:prony}
A clean partial discharge signal can be regarded as a finite combination of
damped complex exponentials. Under this assumption, the so-called
Prony's method allows for the estimation of frequency, amplitude,
phase and damping components of the signal
\cite{Kumaresan.Tufts1982,Hauer.Demeure.ea1990,Kumaresan.Tufts.ea1984}.

Let's use some signals from the dataset (figure
\ref{fig:pd_signal}). The signals contain zeros at the beginning and
at the end. The \texttt{no0} function can remove these parts (figure \ref{fig:no0}).
\begin{Schunk}
\begin{Sinput}
> signals <- lista[1:25]
\end{Sinput}
\end{Schunk}

\begin{figure}
  \centering
\begin{Schunk}
\begin{Sinput}
> xyplot(signals, y.same = NA, FUN = function(x) {
+     xyplot(ts(x))
+ })
\end{Sinput}
\end{Schunk}
\includegraphics{pdCluster-004}
  \caption{Partial discharge signals}
  \label{fig:pd_signal}
\end{figure}


\begin{figure}
  \centering
\begin{Schunk}
\begin{Sinput}
> xyplot(signals, y.same = NA, FUN = function(x) {
+     xyplot(ts(no0(x)))
+ })
\end{Sinput}
\end{Schunk}
\includegraphics{pdCluster-005}
  \caption{Partial discharge signals after a threshold cleaning}
  \label{fig:no0}
\end{figure}

With these cleaned signals the Prony's method can provide their
components (figure \ref{fig:prony}). Since the number of components
must be fixed \emph{a priori}, the function \texttt{compProny} allows
the comparison of different numbers (figure \ref{fig:pronyComp})

\begin{figure}
  \centering
\begin{Schunk}
\begin{Sinput}
> signal <- signals[[3]]
> pr <- prony(signal, M = 10)
> print(xyplot(pr))