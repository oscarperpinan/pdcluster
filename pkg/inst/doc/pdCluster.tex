\documentclass{article}
\usepackage[T1]{fontenc}
\usepackage[utf8]{inputenc}
\usepackage[a4paper]{geometry}
\usepackage{graphicx}
\geometry{verbose,tmargin=2cm,bmargin=2cm,lmargin=1.5cm,rmargin=1.5cm}
\usepackage[usenames,dvipsnames]{xcolor}
\usepackage{url}
\usepackage{hyperref}
\hypersetup{
    bookmarks=true,         % show bookmarks bar?
    unicode=true,          % non-Latin characters in Acrobat’s bookmarks
    bookmarksnumbered=false,
    bookmarksopen=false,
    breaklinks=true,
    backref=true,
    pdftoolbar=true,        % show Acrobat’s toolbar?
    pdfmenubar=true,        % show Acrobat’s menu?
    pdffitwindow=false,     % window fit to page when opened
    pdfstartview={FitH},    % fits the width of the page to the window
    pdftitle={Partial discharge analysis and clustering}
    pdfauthor={Oscar Perpiñán Lamigueiro},     % author
    pdfsubject={Partial discharge clustering},   % subject of the document
    pdfcreator={AucTeX/Emacs},   % creator of the document
    pdfproducer={LaTeX}, % producer of the document
    pdfkeywords={partial discharge, clustering, feature generation,
      graphical tools}, % list of keywords
    pdfnewwindow=true,      % links in new window
    pdfborder={0 0 0},
    colorlinks=true,       % false: boxed links; true: colored links
    linkcolor=BrickRed,          % color of internal links
    citecolor=BrickRed,        % color of links to bibliography
    filecolor=black,      % color of file links
    urlcolor=Blue           % color of external links 
}

\makeatletter
%%%%%%%%%%%%%%%%%%%%%%%%%%%%%% Textclass specific LaTeX commands.
\usepackage[noae]{Sweave}
\newcommand{\Rcode}[1]{{\texttt{#1}}}
\newcommand{\Robject}[1]{{\texttt{#1}}}
\newcommand{\Rcommand}[1]{{\texttt{#1}}}
\newcommand{\Rfunction}[1]{{\texttt{#1}}}
\newcommand{\Rfunarg}[1]{{\textit{#1}}}
\newcommand{\Rpackage}[1]{{\textit{#1}}}
\newcommand{\Rmethod}[1]{{\textit{#1}}}
\newcommand{\Rclass}[1]{{\textit{#1}}}

%%%%%%%%%%%%%%%%%%%%%%%%%%%%%% User specified LaTeX commands.
%\VignetteIndexEntry{Partial Discharges Clustering}
\usepackage{flafter}
\usepackage{boxedminipage}
\renewenvironment{Schunk}{\begin{center}
    \scriptsize
    \begin{boxedminipage}{0.95\textwidth}}{
    \end{boxedminipage}\end{center}}

\usepackage{siunitx}
\sisetup{per=fraction,fraction=nice, decimalsymbol=comma}
%\usepackage{lscape}
\usepackage{mathpazo}%Letra palatino con fuentes para matemáticas

\makeatother

\begin{document}

\setkeys{Gin}{width=0.5\textwidth}

\title{\texttt{pdCluster}: Partial Discharges Clustering} 


\author{Oscar Perpiñán Lamigueiro \and Miguel Angel Sánchez Urán}


\date{17 May 2011}

\maketitle

\begin{Schunk}
\begin{Sinput}
> library(pdCluster)
\end{Sinput}
\end{Schunk}

The set of examples will use a dataset which is loaded with:
\begin{Schunk}
\begin{Sinput}
> load("~/Investigacion/PD/Datos/20100922/DescargasRAW.RData")
\end{Sinput}
\end{Schunk}
\section{Feature generation}

\subsection{Prony}
\label{sec:prony}
A clean partial discharge signal can be regarded as a finite combination of
damped complex exponentials. Under this assumption, the so-called
Prony's method allows for the estimation of frequency, amplitude,
phase and damping components of the signal
\cite{Kumaresan.Tufts1982,Hauer.Demeure.ea1990,Kumaresan.Tufts.ea1984}.

Let's use some signals from the dataset (figure
\ref{fig:pd_signal}). The signals contain zeros at the beginning and
at the end. The \texttt{no0} function can remove these parts (figure \ref{fig:no0}).
\begin{Schunk}
\begin{Sinput}
> signals <- lista[1:25]
\end{Sinput}
\end{Schunk}

\begin{figure}
  \centering
\begin{Schunk}
\begin{Sinput}
> p <- xyplot(signals, y.same = NA, FUN = function(x) {
+     xyplot(ts(x))
+ })
> print(p)
\end{Sinput}
\end{Schunk}
\includegraphics{pdCluster-004}
  \caption{Partial discharge signals}
  \label{fig:pd_signal}
\end{figure}


\begin{figure}
  \centering
\begin{Schunk}
\begin{Sinput}
> p <- xyplot(signals, y.same = NA, FUN = function(x) {
+     xyplot(ts(no0(x)))
+ })
> print(p)
\end{Sinput}
\end{Schunk}
\includegraphics{pdCluster-005}
  \caption{Partial discharge signals after a threshold cleaning}
  \label{fig:no0}
\end{figure}

With these cleaned signals the Prony's method can provide their
components (figure \ref{fig:prony}). Since the number of components
must be fixed \emph{a priori}, the function \texttt{compProny} allows
the comparison of different numbers (figure \ref{fig:pronyComp})

\begin{figure}
  \centering
\begin{Schunk}
\begin{Sinput}
> signal <- signals[[3]]
> pr <- prony(signal, M = 10)
> print(xyplot(pr))
\end{Sinput}
\end{Schunk}
\includegraphics{pdCluster-006}
  \caption{Prony's method results}
  \label{fig:prony}
\end{figure}

\begin{figure}
  \centering
\begin{Schunk}
\begin{Sinput}
> p <- compProny(signal, M = c(10, 20, 30, 40))
> print(p)
\end{Sinput}
\end{Schunk}
\includegraphics{pdCluster-007}
  \caption{Comparison of different Prony decompositions}
  \label{fig:pronyComp}
\end{figure}
 
\subsection{Feature generation}
\label{sec:features}

The \texttt{pdCluster} includes several functions for feature
generation. The \texttt{analysis} functions comprises all of them. The
results for our example signal are:
\begin{Schunk}
\begin{Sinput}
> analysis(signal)
\end{Sinput}
\begin{Soutput}
  RefMax           W1          W2          W3        W4        range   N
1    154 1.585665e-07 2.80655e-05 0.004660942 0.2967786 0.0009498277 323
        energy        nZC   freq1   freq2   damp1   damp2
1 8.301006e-06 0.04643963 3071675 1704751 1140735 3385223
\end{Soutput}
\end{Schunk}

This function can be used with a list of signals in order to obtain a
matrix of features:
\begin{Schunk}
\begin{Sinput}
> analysisList <- lapply(lista[1:10], analysis)
> pdData <- do.call(rbind, analysisList)
\end{Sinput}
\end{Schunk}

Now we need the angle and reflection information, available from
another different file. In order to safely share the information, both
data frames must be reordered by their energy values: 
\begin{Schunk}
\begin{Sinput}
> pdSummary <- read.csv("~/Investigacion/PD/Datos/20100922/descargas.csv")[1:10, 
+     ]
> idxOrderSummary = order(pdSummary$sumaCuadrados)
> idxOrderData = order(pdData$energy)
> pdDataOrdered = cbind(pdData[idxOrderData, ], pdSummary[idxOrderSummary, 
+     c("angulo", "separacionOriginal")])
\end{Sinput}
\end{Schunk}

Later, the data frame to be used with the clustering algorithm has to
ordered by time. Thus the samples of the \texttt{clara} method will
be random.
\begin{Schunk}
\begin{Sinput}
> idx <- do.call(order, pdSummary[idxOrderSummary, c("segundo", 
+     "inicio")])
> pdDataOrdered <- pdDataOrdered[idx, ]
\end{Sinput}
\end{Schunk}

We can now construct a \texttt{PD} object\footnote{The
  \texttt{pdCluster} package is designed with S4 classes and
  methods. Two classes have been defined: \texttt{PD} and \texttt{PDCluster}.}.
\begin{Schunk}
\begin{Sinput}
> pd <- df2PD(pdDataOrdered)
> pd
\end{Sinput}
\begin{Soutput}
Object of class  PD 

Source of measurements:  
Number of observations:  10 
Filtered?:  FALSE 
Filter: `<undef>`()
Transformed?: FALSE 

Data:
     RefMax             W1                  W2                  W3           
 Min.   :  10.0   Min.   :8.417e-14   Min.   :1.487e-11   Min.   :2.875e-09  
 1st Qu.:  56.5   1st Qu.:3.521e-08   1st Qu.:6.808e-06   1st Qu.:1.158e-03  
 Median : 176.0   Median :2.128e-07   Median :2.981e-05   Median :4.565e-03  
 Mean   : 320.0   Mean   :2.423e-04   Mean   :6.606e-03   Mean   :3.205e-02  
 3rd Qu.: 212.0   3rd Qu.:5.609e-07   3rd Qu.:9.860e-05   3rd Qu.:9.841e-03  
 Max.   :1152.0   Max.   :2.234e-03   Max.   :6.465e-02   Max.   :2.749e-01  
       W4                range                 N              energy         
 Min.   :0.0000000   Min.   :8.272e-10   Min.   :  86.0   Min.   :9.544e-18  
 1st Qu.:0.0007306   1st Qu.:2.792e-04   1st Qu.: 230.8   1st Qu.:2.038e-06  
 Median :0.2051125   Median :1.428e-03   Median : 328.0   Median :1.208e-05  
 Mean   :0.1893751   Mean   :7.392e-03   Mean   : 594.7   Mean   :6.944e-04  
 3rd Qu.:0.3037326   3rd Qu.:1.021e-02   3rd Qu.: 374.5   3rd Qu.:6.760e-04  
 Max.   :0.5386891   Max.   :3.070e-02   Max.   :1954.0   Max.   :3.282e-03  
      nZC               freq1             damp1        
 Min.   :0.005249   Min.   : 353312   Min.   : 175314  
 1st Qu.:0.020226   1st Qu.:1370276   1st Qu.: 679159  
 Median :0.030888   Median :2731837   Median :2755076  
 Mean   :0.033111   Mean   :2210122   Mean   :2320301  
 3rd Qu.:0.044445   3rd Qu.:3066133   3rd Qu.:3635957  
 Max.   :0.070093   Max.   :3113590   Max.   :4715804  
Number of reflections:  6 
\end{Soutput}
\end{Schunk}

The results of \texttt{analysis} to the whole dataset are available with:
\begin{Schunk}
\begin{Sinput}
> load("~/Investigacion/PD/Datos/20100922/dfHibr17112010.RData")
> dfHibr <- df2PD(dfHibr)
> dfHibr
\end{Sinput}
\begin{Soutput}
Object of class  PD 

Source of measurements:  
Number of observations:  9955 
Filtered?:  FALSE 
Filter: `<undef>`()
Transformed?: FALSE 

Data:
     RefMax              W1                  W2                  W3           
 Min.   :    0.0   Min.   :0.000e+00   Min.   :0.000e+00   Min.   :0.000e+00  
 1st Qu.:   57.0   1st Qu.:6.838e-14   1st Qu.:1.228e-11   1st Qu.:2.378e-09  
 Median :  205.0   Median :8.011e-13   Median :1.250e-10   Median :1.730e-08  
 Mean   :  559.4   Mean   :1.197e-03   Mean   :2.366e-03   Mean   :1.352e-02  
 3rd Qu.:  896.0   3rd Qu.:1.104e-07   3rd Qu.:1.740e-05   3rd Qu.:2.009e-03  
 Max.   :34079.0   Max.   :2.973e-01   Max.   :1.165e-01   Max.   :5.869e-01  
       W4                range                 N             energy         
 Min.   :0.000e+00   Min.   :0.000e+00   Min.   :    0   Min.   :0.000e+00  
 1st Qu.:3.341e-07   1st Qu.:2.265e-05   1st Qu.:  172   1st Qu.:2.833e-08  
 Median :1.103e-06   Median :1.064e-04   Median :  417   Median :5.665e-07  
 Mean   :7.648e-02   Mean   :5.624e-03   Mean   :  992   Mean   :4.023e-02  
 3rd Qu.:3.686e-02   3rd Qu.:1.026e-03   3rd Qu.: 1568   3rd Qu.:1.307e-05  
 Max.   :1.430e+00   Max.   :3.108e-01   Max.   :34234   Max.   :5.459e+00  
      nZC               freq1              damp1          
 Min.   :0.000000   Min.   :       0   Min.   :        0  
 1st Qu.:0.003629   1st Qu.:  299379   1st Qu.:   178107  
 Median :0.005682   Median :  384840   Median :   427577  
 Mean   :0.013704   Mean   : 1250969   Mean   :  1807198  
 3rd Qu.:0.021283   3rd Qu.: 1849583   3rd Qu.:  1625622  
 Max.   :0.147059   Max.   :50000000   Max.   :205053894  
Number of reflections:  5626 
\end{Soutput}
\end{Schunk}
\section{Transformations}
\label{sec:transform}

Prior to the clustering algorithm, the feature matrix has to be
filtered:

\begin{Schunk}
\begin{Sinput}
> dfFilter <- filterPD(dfHibr)
\end{Sinput}
\end{Schunk}
and transformed \cite{Box.Cox1964}:

\begin{Schunk}
\begin{Sinput}
> dfTrans <- transformPD(dfFilter)
> dfTrans
\end{Sinput}
\begin{Soutput}
Object of class  PD 

Source of measurements:  
Number of observations:  3695 
Filtered?:  TRUE 
Filter: N > 10 & nZC > 0 & W4 > 0 & freq1 > 1 & freq1 < 2e+07 & !refl
Transformed?: TRUE 

Data:
     RefMax             W1                W2                W3          
 Min.   : 0.000   Min.   :-53.341   Min.   :-38.720   Min.   :-27.9936  
 1st Qu.: 6.809   1st Qu.:-45.027   1st Qu.:-29.386   1st Qu.:-20.6000  
 Median : 7.896   Median :-26.477   Median :-18.205   Median :-13.0398  
 Mean   : 8.387   Mean   :-30.009   Mean   :-19.236   Mean   :-12.7681  
 3rd Qu.:10.407   3rd Qu.:-18.402   3rd Qu.:-10.554   3rd Qu.: -5.1754  
 Max.   :18.979   Max.   : -1.496   Max.   : -2.249   Max.   : -0.5339  
       W4               range               N             energy       
 Min.   :-20.5419   Min.   :-31.952   Min.   :3.381   Min.   :-59.712  
 1st Qu.:-14.3313   1st Qu.:-12.589   1st Qu.:3.934   1st Qu.:-19.287  
 Median : -7.3826   Median :-10.500   Median :4.171   Median :-16.453  
 Mean   : -8.1065   Mean   : -9.551   Mean   :4.217   Mean   :-14.776  
 3rd Qu.: -1.2579   3rd Qu.: -6.206   3rd Qu.:4.561   3rd Qu.: -8.933  
 Max.   :  0.1203   Max.   : -1.210   Max.   :5.339   Max.   :  1.650  
      nZC              freq1           damp1       
 Min.   :-12.686   Min.   :4.996   Min.   : 6.581  
 1st Qu.: -9.468   1st Qu.:5.131   1st Qu.:10.800  
 Median : -8.107   Median :5.226   Median :11.327  
 Mean   : -7.424   Mean   :5.216   Mean   :11.527  
 3rd Qu.: -5.080   3rd Qu.:5.313   3rd Qu.:12.434  
 Max.   : -3.268   Max.   :5.433   Max.   :14.677  
Number of reflections:  0 
\end{Soutput}
\end{Schunk}

The figure \ref{fig:histogram} compares the datasets after and before
of the transformations:

\begin{Schunk}
\begin{Sinput}
> nZCbefore <- as.data.frame(dfFilter)$nZC
> nZCafter <- as.data.frame(dfTrans)$nZC
> comp <- data.frame(After = nZCafter, Before = nZCbefore)
\end{Sinput}
\end{Schunk}

\begin{figure}
  \centering
\begin{Schunk}
\begin{Sinput}
> h <- histogram(~After + Before, data = comp, scales = list(x = list(relation = "free"), 
+     y = list(relation = "free", draw = FALSE)), breaks = 100, 
+     col = "gray", xlab = "", strip.names = c(TRUE, TRUE), bg = "gray", 
+     fg = "darkblue")
> print(h)
\end{Sinput}
\end{Schunk}
\includegraphics{pdCluster-017}
  \caption{Histogram of a collection of partial discharges}
  \label{fig:histogram}
\end{figure}

The \texttt{filterPD} method is a wrapper for the general
\texttt{subset} method. With \texttt{subset} it is possible to extract
a group of samples based on a condition and select only certain
columns. The figure \ref{fig:splom_subset} shows a scatterplot matrix
with \texttt{splom} of this subset. The \texttt{splom} graphical tool
is explained in the next section.
\begin{Schunk}
\begin{Sinput}
> dfTransSubset <- subset(dfTrans, subset = (angle >= 90 & angle <= 
+     180), select = c(energy, W1, nZC))
> dfTransSubset
\end{Sinput}
\begin{Soutput}
Object of class  PD 

Source of measurements:  
Number of observations:  624 
Filtered?:  TRUE 
Filter: N > 10 & nZC > 0 & W4 > 0 & freq1 > 1 & freq1 < 2e+07 & !refl
Transformed?: TRUE 

Data:
     energy              W1               nZC         
 Min.   :-42.785   Min.   :-52.106   Min.   :-12.154  
 1st Qu.:-19.589   1st Qu.:-45.646   1st Qu.: -9.568  
 Median :-17.837   Median :-41.668   Median : -9.184  
 Mean   :-15.985   Mean   :-34.488   Mean   : -8.000  
 3rd Qu.:-12.644   3rd Qu.:-20.409   3rd Qu.: -6.064  
 Max.   :  1.650   Max.   : -1.743   Max.   : -3.896  
Number of reflections:  0 
\end{Soutput}
\end{Schunk}

\begin{figure}
  \centering
\begin{Schunk}
\begin{Sinput}
> p <- splom(dfTransSubset)
> print(p)
\end{Sinput}
\end{Schunk}
\includegraphics{pdCluster-019}
  \caption{Scatterplot matrix of a subset of \texttt{dfTrans}.}
  \label{fig:splom_subset}
\end{figure}


\subsection{Graphical tools}
\label{sec:graphics}

The \texttt{pdCluster} packages includes a set of graphical
exploratory tools, such as a scatterplot matrices with hexagonal
binning \cite{Carr.Littlefield.ea1987} (figure \ref{fig:splom}), density plots (figure
\ref{fig:density}), histograms (figure \ref{fig:histograms}) or phase
resolved partial discharge patterns, both with partial transparency
(figure \ref{fig:xyplot}) or hexagonal binning (figure
\ref{fig:hexbinplot}).

\begin{figure}
  \centering
\begin{Schunk}
\begin{Sinput}
> p <- splom(dfTrans)
> print(p)
\end{Sinput}
\end{Schunk}
\includegraphics{pdCluster-020}
  \caption{Scatterplot matrix of a collection of partial discharges}
  \label{fig:splom}
\end{figure}

\begin{figure}
  \centering
\begin{Schunk}
\begin{Sinput}
> p <- densityplot(dfTrans)
> print(p)
\end{Sinput}
\end{Schunk}
\includegraphics{pdCluster-021}
  \caption{Density plot of a collection of partial discharges}
  \label{fig:density}
\end{figure}

\begin{figure}
  \centering
\begin{Schunk}
\begin{Sinput}
> p <- histogram(dfTrans)
> print(p)
\end{Sinput}
\end{Schunk}
\includegraphics{pdCluster-022}
  \caption{Histograms of a collection of partial discharges}
  \label{fig:histograms}
\end{figure}

\begin{figure}
  \centering
\begin{Schunk}
\begin{Sinput}
> p <- xyplot(dfTrans)
> print(p)
\end{Sinput}
\end{Schunk}
\includegraphics{pdCluster-023}
  \caption{Partial discharge phase resolved pattern.}
  \label{fig:xyplot}
\end{figure}
 

\begin{figure}
  \centering
\begin{Schunk}
\begin{Sinput}
> p <- hexbinplot(dfTrans)
> print(p)
\end{Sinput}
\end{Schunk}
\includegraphics{pdCluster-024}
  \caption{Partial discharge phase resolved pattern with hexbinplot }
  \label{fig:hexbinplot}
\end{figure}



\section{Clustering}
\label{sec:clustering}

The filtered and transformed object can now be used with the
clustering algorithm \cite{Struyf.Hubert.ea1997}. The results are
displayed with a phase resolved pattern with clusters in separate
panels in the figure \ref{fig:xyplotClusterSeparate}. The colors
encode the distance of each point to the \emph{medoid} of its
cluster. The figure \ref{fig:xyplotCluster} displays the same pattern
with superposed clusters. Here the colors encode the membership to a
certain cluster, and transparency is used to denote density of points
in a region.

The results can be easily understood with the density plots of each
cluster and feature (figure \ref{fig:densityCluster}) or with the
histograms (figure \ref{fig:histogramsCluster}).

\begin{Schunk}
\begin{Sinput}
> dfTransCluster <- claraPD(dfTrans, noise.level = 0.7, noise.rm = TRUE)
> dfTransCluster
\end{Sinput}
\begin{Soutput}
Object of class  PD 

Source of measurements:  
Number of observations:  2643 
Filtered?:  TRUE 
Filter: N > 10 & nZC > 0 & W4 > 0 & freq1 > 1 & freq1 < 2e+07 & !refl
Transformed?: TRUE 

Data:
     RefMax             W1                W2                W3         
 Min.   : 1.972   Min.   :-53.341   Min.   :-38.720   Min.   :-27.994  
 1st Qu.: 7.081   1st Qu.:-45.517   1st Qu.:-29.706   1st Qu.:-20.887  
 Median : 9.025   Median :-35.255   Median :-24.944   Median :-19.066  
 Mean   : 8.841   Mean   :-31.585   Mean   :-20.194   Mean   :-13.379  
 3rd Qu.:10.612   3rd Qu.:-18.438   3rd Qu.:-10.554   3rd Qu.: -5.171  
 Max.   :16.753   Max.   : -6.286   Max.   : -2.715   Max.   : -1.162  
       W4                range               N             energy       
 Min.   :-20.54193   Min.   :-18.600   Min.   :3.381   Min.   :-30.428  
 1st Qu.:-14.48318   1st Qu.:-12.492   1st Qu.:3.943   1st Qu.:-19.013  
 Median :-13.66034   Median :-10.973   Median :4.421   Median :-16.647  
 Mean   : -8.43715   Mean   : -9.288   Mean   :4.272   Mean   :-14.145  
 3rd Qu.: -1.23210   3rd Qu.: -4.736   3rd Qu.:4.569   3rd Qu.: -7.572  
 Max.   : -0.09173   Max.   : -1.210   Max.   :5.202   Max.   :  1.650  
      nZC              freq1           damp1       
 Min.   :-12.192   Min.   :5.005   Min.   : 6.581  
 1st Qu.: -9.523   1st Qu.:5.131   1st Qu.:10.706  
 Median : -8.822   Median :5.139   Median :11.135  
 Mean   : -7.414   Mean   :5.211   Mean   :11.460  
 3rd Qu.: -4.954   3rd Qu.:5.312   3rd Qu.:12.475  
 Max.   : -3.268   Max.   :5.329   Max.   :13.115  
Number of reflections:  0 
Number of clusters:  7 
Number of elements per cluster:

   1    2    3    4 
1383  696  420  144 

Metric:  manhattan 
Number of simulations:  25 
Noise level:  0.7 
Distances 
      dist            distRel         distFactor     
 Min.   : 0.1020   Min.   :0.0000   Min.   :  1.000  
 1st Qu.: 2.1595   1st Qu.:0.7387   1st Qu.:  3.000  
 Median : 4.9140   Median :0.8760   Median :  5.000  
 Mean   : 6.8265   Mean   :0.8294   Mean   :  4.977  
 3rd Qu.:11.0155   3rd Qu.:0.9463   3rd Qu.:  7.000  
 Max.   :50.0681   Max.   :0.9975   Max.   :  9.000  
                                    NA's   :263.000  
\end{Soutput}
\end{Schunk}

\begin{figure}
  \centering
\begin{Schunk}
\begin{Sinput}
> p <- xyplot(dfTransCluster)
> print(p)
\end{Sinput}
\end{Schunk}
\includegraphics{pdCluster-026}
  \caption{Partial discharge phase resolved pattern with clusters in
    separate panels.}
  \label{fig:xyplotClusterSeparate}
\end{figure}

\begin{figure}
  \centering
\begin{Schunk}
\begin{Sinput}
> p <- xyplot(dfTransCluster, panelClust = FALSE)
> print(p)
\end{Sinput}
\end{Schunk}
\includegraphics{pdCluster-027}
  \caption{Partial discharge phase resolved pattern with clusters
    marked with colors.}
  \label{fig:xyplotCluster}
\end{figure}

\begin{figure}
  \centering
\begin{Schunk}
\begin{Sinput}
> p <- densityplot(dfTransCluster)
> print(p)
\end{Sinput}
\end{Schunk}
\includegraphics{pdCluster-028}
  \caption{Density plot of the clusters of partial discharges}
  \label{fig:densityCluster}
\end{figure}

\begin{figure}
  \centering
\begin{Schunk}
\begin{Sinput}
> p <- histogram(dfTransCluster)
> print(p)
\end{Sinput}
\end{Schunk}
\includegraphics{pdCluster-029}
  \caption{Histograms of the clusters of partial discharges}
  \label{fig:histogramsCluster}
\end{figure}

\bibliographystyle{alpha}
\bibliography{/home/oscar/Bibliografia/BibUTF8}


\end{document}
